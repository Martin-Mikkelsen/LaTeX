\documentclass{article}
\usepackage[english, danish]{babel}
\usepackage[utf8]{inputenc}
\usepackage{amsmath}
\usepackage{graphicx}
\usepackage{amssymb}
\usepackage{fancyhdr}
\usepackage{kpfonts}
\usepackage{color} 
\usepackage{fancyvrb} 
\usepackage{lastpage}
\usepackage{pgf,tikz,pgfplots}
\usepackage{mathrsfs}
\usetikzlibrary{arrows}
\usepackage[numbered,framed]{matlab-prettifier}
\fancypagestyle{firststyle}
{\fancyhf{}
	\fancyhead[R]{Fysik hold 1}
	\fancyhead[L]{Aarhus Universitet}
	\fancyfoot[C]{\footnotesize Side \thepage\ af \pageref{LastPage}}
	\renewcommand{\headrulewidth}{1pt}}
\pagestyle{fancy}
\fancyhf{} % clear all header and footer fields
\fancyfoot[C]{\footnotesize Side \thepage\ af \pageref{LastPage}}
\renewcommand{\headrulewidth}{0pt}
\newcommand{\N}{\mathbb{N}}
\newcommand{\Z}{\mathbb{Z}}
\newcommand{\Q}{\mathbb{Q}}
\newcommand{\R}{\mathbb{R}}
\newcommand{\C}{\mathbb{C}}
\newcommand{\F}{\mathbb{F}}
\newcommand{\V}{\mathcal{V}}
\newcommand{\eq}[1]{\begin{align*}#1\end{align*}}
\newcommand{\pmat}[4]{\begin{pmatrix} #1 & #2 \\ #3 & #4 \end{pmatrix}}
\newcommand{\pvec}[4]{\begin{pmatrix} #1   \\ #2 \\ #3 \\ #4\end{pmatrix}}



\author{Martin Mikkelsen \\ \\
Studienummer: 201706771} 
\title{Afleveringsopgave 1}

\begin{document}
\maketitle
\thispagestyle{firststyle}
\section*{Opgave 1} 
Beregn $ (4+2i)^2, |4+2i|^2, \frac{(3i-3)^2}{i}, \frac{(i-1)}{i}(1+i) $ 
\eq{(4+2i)^2=(4+2i)(4+2i)=12+16i}
\section*{Opgave 1a} 
$ |z|=zz^* $
\eq{(4+2i)(4-2i)=16-4i^2=20}
\section*{Opgave 1b}
\eq{\frac{(3i-3)^2}{i}=-i(3i-3)^2=-i((3i-3)(3i-3))=-i(9i^2-9i-9i+9)=-i(-18)=-18}
\section*{Opgave 1c}
\eq{\frac{(i-1)}{i}(1+i)=-i((i-1)(1+i))=-i(i+i^2-1-i)=-i(i-2-i)=2i}
\section*{Opgave 2a}
\eq{\frac{a}{x}=\frac{b}{c} \Rightarrow \frac{1}{x}=\frac{b}{ac} \Rightarrow x= \frac{ac}{b}} 
\section*{Opgave 2b}
\eq{\frac{a}{x}+\frac{b}{c}=1 \Rightarrow \frac{a}{x}=1-\frac{b}{c} \Rightarrow a=(1-\frac{b}{c})x \Rightarrow \frac{a}{1-\frac{b}{c}}=x} 
\section*{Opgave 3}
\eq{f(x)=sin(\alpha x)sin(kx)} 
\eq{f'(x)=\alpha cos(\alpha x)sin(kx)+sin(\alpha x)kcos(kx) }
\section*{Opgave 4}
\begin{figure}[!htb]
	\minipage{0.32\textwidth}
	\includegraphics[width=\linewidth]{1.png}
	\caption{$ \frac{1}{2}e^{-\frac{x}{2}} $}
	\endminipage\hfill
	\minipage{0.32\textwidth}
	\includegraphics[width=\linewidth]{2.png}
	\caption{$ \frac{1}{2}e^{-2x} $}
	\endminipage\hfill
	\minipage{0.32\textwidth}%
	\includegraphics[width=\linewidth]{3.png}
	\caption{$ e^{-x^2} $}
	\endminipage
\end{figure}

\section*{Opgave 5a}
\eq{\int_{0}^{\infty} e^{-\frac{x}{10}} dx}
\eq{\int e^{-\frac{x}{10}} dx=-10 e^{-\frac{x}{10}}}
\eq{\int_{0}^{\infty} e^{-\frac{x}{10}} dx=-10 \cdot 0 - (-10)\cdot e^0=10}
\section*{Opgave 5b}
\eq{\int_{0}^{\pi} sin(x) dx}
\eq{\int sin(x) dx = -cos(x) dx}
\eq{\int_{0}^{\pi} sin(x) dx = -cos(\pi)-(-cos(x))=2}
\section*{Opgave 6}
\eq{f(x)=Asin(kx)+Bcos(kx)}
\eq{f'(x)=Akcos(kx)-Bksin(kx)}
\eq{f''(x)=-Ak^2sin(kx)-Bk^2cos(kx)}
\eq{-k^2\cdot f(x)=-Ak^2sin(kx)-Bk^2cos(kx)}
$ f''(x)$ og $ -k^2 \cdot f(x)$ er lig hinanden og funktionen er løsning.
\end{document}  
